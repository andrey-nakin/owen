\documentclass[12pt, a4paper]{article}
\usepackage[utf8]{inputenc}
\usepackage[russian]{babel}
\usepackage{hyperref}

\newcommand{\CMD}[1]{{\tt \hyperref[#1]{#1}}}
\newcommand{\SEEALSO}{\subsubsection*{См. также}}
\newcommand{\EXAMPLE}{\subsubsection*{Пример}}
\newcommand{\PURPOSE}{\subsubsection*{Назначение}}
\newcommand{\SYNTAX}[1]{\subsubsection*{Синтакс}{\tt #1}}
\newcommand{\NOTES}{\subsubsection*{Примечание}}
\newcommand{\ARGUMENTS}{\subsubsection*{Аргументы}}
\newcommand{\RETURN}{\subsubsection*{Возвращаемое значение}}
\newcommand{\NORETURN}{\subsubsection*{Возвращаемое значение} Отсутствует}
\newcommand{\COMMAND}[1]{\subsection{#1}\label{#1}}
\newcommand{\BEGINARGUMENTS}{\ARGUMENTS\begin{itemize}}
\newcommand{\ENDARGUMENTS}{\end{itemize}}
\newcommand{\NOARGUMENTS}{\ARGUMENTS Отсутствуют}
\newcommand{\ARGUMENT}[1]{\item {\tt \mbox{#1}}~---}
\newcommand{\DESCARGUMENT}{\ARGUMENT{desc} дескриптор, полученный командой \CMD{::owen::configure}.}

\title{Библиотека OWEN \\ Реализация протокола ОВЕН на языке программирования Tcl \\ Руководство пользователя}

\begin{document}

\maketitle

\tableofcontents

\section{Общие сведения}

Данная библиотека реализует протокол обмена компании ОВЕН на языке программирования Tcl. Данный протокол используется для взаимодействия с устройствами производства данной компании через последовательный порт. Описание протокола можно найти на сайте компании \href{http://owen.ru}{owen.ru}. 

В данной версии библиотека позволяет:

\begin{itemize}
\item отправлять команды на устройство;
\item считывать строковые параметры из устройства;
\item считывать и записывать целочисленные параметры;
\item считывать и записывать параметры в формате FLOAT24 (24-битовое число с плавающей точкой).
\end{itemize}

\section{Описание библиотеки}

\subsection{Дескриптор}

Перед началом обмена с устройством необходимо создать {\it дескриптор} --- структура, описывающая параметры устройства, такие как адрес последовательного порта и его установки, адрес самого устройства, величина таймаута и пр. Дескриптор создаётся командой \CMD{::owen::configure}:

\begin{verbatim} 
# Последовательный порт COM1, адрес устройства 32
set desc [::owen::configure -port COM1 -addr 32]
# Дескриптор создан
# Теперь запрашиваем параметр DEV
set dev [::owen::readString $desc DEV]
\end{verbatim} 

Полученный дескриптор далее используется для обмена.

\subsection{Использование порта}

Последовательный порт, по которому происходит обмен, открывается непосредственно перед передачей данных и закрывается сразу после получения ответа от устройства.

\subsection{Информация об ошибке}

Все команды обмена с устройством, которые возвращают какое-либо значение, в случае ошибки возвращают пустую строку без генерации исключительной ситуации. Чтобы узнать детальную информацию об ошибке, используются команды \CMD{:owen::lastStatus} и \CMD{::owen::lastError}.

Команда \CMD{:owen::lastStatus} возвращает {\it статус} последней выполненной операции. Если статус равен {\tt ::owen::STATUS\_OK}, операция прошла успешно и нет необходимости вызывать команду \CMD{::owen::lastError}.

\begin{itemize}
\item {\tt ::owen::STATUS\_OK} --- операция прошла успешно, не нужно далее вызывать \CMD{::owen::lastError}.

\item {\tt ::owen::STATUS\_EXCEPTION} --- устройство сигнализирует об исключительной ситуации, нужно вызвать команду \CMD{::owen::lastError}, чтобы получить код ситуации. Например, код {\tt 0xFE} говорит об отсутствии связи с АЦП. Расшифровка кодов исключительных ситуаций см. в документации к устройству.

\item {\tt ::owen::STATUS\_NETWORK\_ERROR} --- произошла ошибка обмена данными. Как правило, данный статус говорит о попытке получить или записать значение параметра с неверным типом значения. Например, целочисленный параметр считывается командой \CMD{::owen::readFloat24}, или 16-битовое целочисленное значение записывается командой \CMD{::owen::writeInt8}. В этом случае необходимо вызвать команду \CMD{::owen::lastError}, чтобы узнать код ошибки. Список кодов ошибок приведён в документации к устройству. Например, код {\tt 0x31} соответствует ошибке <<Размер поля данных не соответствует ожидаемому>>. Также имеется несколько дополнительных кодов ошибок:

\begin{itemize}
\item {\tt ::owen::ERROR\_BAD\_DATA} --- полученные данные не являются корректным пакетом. Вероятно требуется уменьшить скорость обмена.
\item {\tt ::owen::ERROR\_BAD\_LENGTH} --- полученный пакет имеет неверную длину. Нужно проверить правильность работы устройства.
\item {\tt ::owen::ERROR\_TIMEOUT} --- таймаут ожидания ответа от устройства. Необходимо проверить правильность физического подключения и питание устройства.
\end{itemize}

\item {\tt ::owen::STATUS\_PORT\_ERROR} --- произошла ошибка открытия или конфигурирования последовательного порта. Команда \CMD{::owen::lastError} вернёт текстовое описание ошибки. Например, данная ситуация может произойти при попытке соединения с устройством по отсутствующему порту.
\end{itemize}

Команда \CMD{::owen::lastError} возвращает детальное описание ошибки. Трактовки значения зависит от статуса, см. выше. В зависимости от статуса команда возвращает либо числовой код, либо текстовое описание ошибки.

Информация об ошибке сохраняется в глобальной переменной, поэтому если производится работа с несколькими устройствами одновременно, нужно сохранять статус и ошибку в локальных переменных сразу после операции над каждый устройством.

\section{Описание команд}

\COMMAND{::owen::configure}

\PURPOSE

Создаёт дескриптор устройства по заданным параметрам. Не производит открытие или проверку существования последовательного порта.

\SYNTAX{::owen::configure args}

\BEGINARGUMENTS
\ARGUMENT{args} Параметры устройства в виде пар <<параметр>>-<<значение>>. Список возможных параметров:

\begin{itemize}
\item {\tt -port} --- имя последовательного порта, по которому происходит обмен данными, например, <<{\tt COM1}>>.
\item {\tt -settings} --- параметры последовательного порта в виде строки, например, <<{\tt 9600,8,n,1}>>.
\item {\tt -timeout} --- таймаут ожидания ответа от устройства в мс, например, <<{\tt 500}>>.
\item {\tt -numOfAttempts} --- число попыток связи с устройством в случае невозможности открыть порт или помех на линии, например, <<{\tt 3}>>.
\item {\tt -addr} --- адрес устройства, например, <<{\tt 16}>>.
\item {\tt -addrType} --- длина адреса: {\tt 0} --- 8 бит, {\tt 1} --- 11 бит. Можно использовать предопределённые константы {\tt ::owen::ADDR\_TYPE\_8BIT} и {\tt ::owen::ADDR\_TYPE\_11BIT}.
\end{itemize}

\ENDARGUMENTS

\RETURN

Дескриптор устройства.

\EXAMPLE

\begin{verbatim} 
# устройство подключено к порту COM1
# и 8-битный адрес 32
set desc [::owen::configure -port COM1 -addr 32 -addrType 0]
# в переменной desc - дескриптор
\end{verbatim} 

\COMMAND{::owen::lastStatus}

\PURPOSE

Возвращает статус последней произведённой операции. Обычно используется в паре с \CMD{::owen::lastError}

\SYNTAX{::owen::lastStatus}

\NOARGUMENTS

\RETURN

Целочисленное число с кодом статуса.

\EXAMPLE

\begin{verbatim}
# создаём дескриптор
set desc [::owen::configure -port COM1 -addr 32 -addrType 0]
# считываем параметр DEV
set dev [::owen::readString $desc DEV]
# проверяем статус
if { [::owen::lastStatus] != $::owen::STATUS_OK } {
  error "Error reading DEV parameter"
}
\end{verbatim} 

\SEEALSO
\CMD{::owen::lastError}

\COMMAND{::owen::lastError}

\PURPOSE

Возвращает описание ошибки от последней произведённой операции. Обычно используется в паре с \CMD{::owen::lastStatus}

\SYNTAX{::owen::lastError}

\NOARGUMENTS

\RETURN

Целочисленное число с кодом или текстовое описание ошибки. Трактовка кода ошибки зависит от статуса.

\EXAMPLE

\begin{verbatim}
# создаём дескриптор
set desc [::owen::configure -port COM1 -addr 32 -addrType 0]
# считываем параметр DEV
set dev [::owen::readString $desc DEV]
# проверяем статус
if { [::owen::lastStatus] != $::owen::STATUS_OK } {
  error "Error [::lastError]"
}
\end{verbatim} 

\SEEALSO
\CMD{::owen::lastStatus}

\COMMAND{::owen::sendCommand}

\PURPOSE

Отправляет команду устройству без ожидания ответа.

\SYNTAX{::owen::sendCommand desc cmd}

\BEGINARGUMENTS
\DESCARGUMENT
\ARGUMENT{cmd} строка с командой.
\ENDARGUMENTS

\NORETURN

\EXAMPLE

\begin{verbatim} 
# создаём дескриптор
set desc [::owen::configure -port COM1 -addr 32]
# перезагрузим устройство
::owen::sendCommand $desc INIT
\end{verbatim} 

\COMMAND{::owen::readString}

\PURPOSE

Считывает строковый параметр из устройства.

\SYNTAX{::owen::readString desc parameter}

\BEGINARGUMENTS
\DESCARGUMENT
\ARGUMENT{parameter} имя параметра.
\ENDARGUMENTS

\RETURN

Значение параметра или пустая строка в случае ошибки.

\EXAMPLE

\begin{verbatim} 
# создаём дескриптор
set desc [::owen::configure -port COM1 -addr 32]
# считывает тип устройства
set dev [::owen::readString $desc DEV]
\end{verbatim} 

\COMMAND{::owen::readInt}

\PURPOSE

Считывает целочисленный параметр из устройства.

\SYNTAX{::owen::readInt desc parameter ?index?}

\BEGINARGUMENTS
\DESCARGUMENT
\ARGUMENT{parameter} имя параметра.
\ARGUMENT{index} индекс параметра. Может отсутствовать, если параметр не имеет индекса.
\ENDARGUMENTS

\RETURN

Значение параметра или пустая строка в случае ошибки.

\EXAMPLE

\begin{verbatim} 
# создаём дескриптор
set desc [::owen::configure -port COM1 -addr 32]
# считываем адрес устройства
set addr [::owen::readInt $desc ADDR]
\end{verbatim} 

\COMMAND{::owen::writeInt8}

\PURPOSE

Записывает целочисленный 8-битный параметр в устройство, после чего считывает и возвращает этот параметр.

\SYNTAX{::owen::writeInt8 desc parameter index value}

\BEGINARGUMENTS
\DESCARGUMENT
\ARGUMENT{parameter} имя параметра.
\ARGUMENT{index} индекс параметра. Если параметр не имеет индекса, нужно указать {\tt -1}.
\ARGUMENT{value} Значение параметра.
\ENDARGUMENTS

\RETURN

Значение параметра или пустая строка в случае ошибки.

\EXAMPLE

\begin{verbatim} 
# создаём дескриптор
set desc [::owen::configure -port COM1 -addr 32]
# время задержки
set rsdl [::owen::writeInt8 $desc RSDL -1 20]
\end{verbatim} 

\COMMAND{::owen::writeInt16}

\PURPOSE

Записывает целочисленный 16-битный параметр в устройство, после чего считывает и возвращает этот параметр.

\SYNTAX{::owen::writeInt8 desc parameter index value}

\BEGINARGUMENTS
\DESCARGUMENT
\ARGUMENT{parameter} имя параметра.
\ARGUMENT{index} индекс параметра. Если параметр не имеет индекса, нужно указать {\tt -1}.
\ARGUMENT{value} Значение параметра.
\ENDARGUMENTS

\RETURN

Значение параметра или пустая строка в случае ошибки.

\EXAMPLE

\begin{verbatim} 
# создаём дескриптор
set desc [::owen::configure -port COM1 -addr 32]
# задаём новый адрес
set addr [::owen::writeInt16 $desc ADDR -1 16]
\end{verbatim} 

\COMMAND{::owen::readFloat24}

\PURPOSE

Считывает числовой параметр в формате FLOAT24 из устройства.

\SYNTAX{::owen::readFloat24 desc parameter ?index?}

\BEGINARGUMENTS
\DESCARGUMENT
\ARGUMENT{parameter} имя параметра.
\ARGUMENT{index} индекс параметра. Может отсутствовать, если параметр не имеет индекса.
\ENDARGUMENTS

\RETURN

Значение параметра или пустая строка в случае ошибки.

\EXAMPLE

\begin{verbatim} 
# создаём дескриптор
set desc [::owen::configure -port COM1 -addr 32]
# считываем уставку
set setPoint [::owen::readFloat24 $desc SP 0]
\end{verbatim} 

\COMMAND{::owen::writeFloat24}

\PURPOSE

Записывает числовой параметр в формате FLOAT24 в устройство, после чего считывает и возвращает этот параметр.

\SYNTAX{::owen::writeFloat24 desc parameter index value}

\BEGINARGUMENTS
\DESCARGUMENT
\ARGUMENT{parameter} имя параметра.
\ARGUMENT{index} индекс параметра. Если параметр не имеет индекса, нужно указать {\tt -1}.
\ARGUMENT{value} Значение параметра.
\ENDARGUMENTS

\RETURN

Значение параметра или пустая строка в случае ошибки.

\EXAMPLE

\begin{verbatim} 
# создаём дескриптор
set desc [::owen::configure -port COM1 -addr 32]
# новая уставка
set setPoint [::owen::writeFloat24 $desc SP 0 123.4]
\end{verbatim} 

\end{document}
