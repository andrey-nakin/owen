\documentclass[12pt, a4paper]{article}
\usepackage[utf8]{inputenc}
\usepackage[russian]{babel}
\usepackage{hyperref}

\newcommand{\CMD}[1]{{\tt \hyperref[#1]{#1}}}

\title{Библиотека OWEN \\ Реализация протокола ОВЕН на языке программирования Tcl \\ Руководство пользователя}

\begin{document}

\maketitle

\section{Общие сведения}

Данная библиотека реализует протокол обмена компании ОВЕН на языке программирования Tcl. Данный протокол используется для взаимодействия с устройствами производства данной компании через последовательный порт. Описание протокола можно найти на сайте компании \href{http://owen.ru}{owen.ru}. 

В данной версии библиотека позволяет:

\begin{itemize}
\item отправлять команды на устройство;
\item считывать строковые параметры из устройства;
\item считывать и записывать целочисленные параметры;
\item считывать и записывать параметры в формате FLOAT24 (24-битовое число с плавающей точкой).
\end{itemize}

\section{Описание библиотеки}

\subsection{Принцип работы}

\subsubsection{Дескриптор}

Перед началом обмена с устройством необходимо создать {\it дескриптор} --- структура, описывающая параметры устройства, такие как адрес последовательного порта и его установки, адрес самого устройства, величина таймаута и пр. Дескриптор создаётся командой \CMD{::owen::configure}:

\begin{verbatim} 
# Последовательный порт COM1, адрес устройства 32
set desc [::owen::configure -port COM1 -addr 32]
# Дескриптор создан
# Теперь запрашиваем параметр DEV
set dev [::owen::readString $desc DEV]
\end{verbatim} 

Полученный дескриптор далее используется для обмена.

\subsubsection{Использование порта}

Последовательный порт, по которому происходит обмен, открывается непосредственно перед передачей данных и закрывается сразу после получения ответа от устройства.

\subsubsection{Информация об ошибке}

Все команды обмена с устройством, которые возвращают какое-либо значение, в случае ошибки возвращают пустую строку без генерации исключительной ситуации. Чтобы узнать детальную информацию об ошибке, используются команды \CMD{:owen::lastStatus} и \CMD{::owen::lastError}.

Команда \CMD{:owen::lastStatus} возвращает {\it статус} последней выполненной операции. Если статус равен {\tt ::owen::STATUS\_OK}, операция прошла успешно и нет необходимости вызывать команду \CMD{::owen::lastError}.

\begin{itemize}
\item {\tt ::owen::STATUS\_OK} --- операция прошла успешно, не нужно далее вызывать \CMD{::owen::lastError}.

\item {\tt ::owen::STATUS\_EXCEPTION} --- устройство сигнализирует об исключительной ситуации, нужно вызвать команду \CMD{::owen::lastError}, чтобы получить код ситуации. Например, код {\tt 0xFE} говорит об отсутствии связи с АЦП. Расшифровка кодов исключительных ситуаций см. в документации к устройству.

\item {\tt ::owen::STATUS\_NETWORK\_ERROR} --- произошла ошибка обмена данными. Как правило, данный статус говорит о попытке получить или записать значение параметра с неверным типом значения. Например, целочисленный параметр считывается командой \CMD{::owen::readFloat24}, или 16-битовое целочисленное значение записывается командой \CMD{::owen::writeInt8}. В этом случае необходимо вызвать команду \CMD{::owen::lastError}, чтобы узнать код ошибки. Список кодов ошибок приведён в документации к устройству. Например, код {\tt 0x31} соответствует ошибке <<Размер поля данных не соответствует ожидаемому>>. Также имеется несколько дополнительных кодов ошибок:

\begin{itemize}
\item {\tt ::owen::ERROR\_BAD\_DATA} --- полученные данные не являются корректным пакетом. Вероятно требуется уменьшить скорость обмена.
\item {\tt ::owen::ERROR\_BAD\_LENGTH} --- полученный пакет имеет неверную длину. Нужно проверить правильность работы устройства.
\item {\tt ::owen::ERROR\_TIMEOUT} --- таймаут ожидания ответа от устройства. Необходимо проверить правильность физического подключения и питание устройства.
\end{itemize}

\item {\tt ::owen::STATUS\_PORT\_ERROR} --- произошла ошибка открытия или конфигурирования последовательного порта. Команда \CMD{::owen::lastError} вернёт текстовое описание ошибки. Например, данная ситуация может произойти при попытке соединения с устройством по отсутствующему порту.
\end{itemize}

Команда \CMD{::owen::lastError} возвращает детальное описание ошибки. Трактовки значения зависит от статуса, см. выше. В зависимости от статуса команда возвращает либо числовой код, либо текстовое описание ошибки.

\end{document}
